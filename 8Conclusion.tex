\chapter{Conclusions}
\label{chapter:conclude}
In this thesis, we study the surface texture processing problem from the perspective of reconstruction and understanding.

The surface texture has a strong relationship with the scanning images via rendering and reconstruction. The texture space is an important 2D parameterization space that is helpful for texture understanding including the semantics and the parameterization itself. Therefore, we reconstruct the high-quality color texture of the 3D surfaces in real environment captured by commodity 3D scanners, build a robust and scalable quadrangulation algorithm for seamless surface parameterization, and provide effective solutions for learning geometric and semantic information from the surface texture. Finally, we find a solution to train a network that predicts canonical parameterization directly from color signals.

For texture reconstruction, we address the inconsistent color mapping problem in the scanning data using traditional color consistency optimization-based methods in \emph{3DLite}, where we propose to compensate all the artifacts by explicitly warping and stitching image fragments from low-quality RGB input data to achieve high-resolution, sharp surface textures. We jointly optimize the sparse and dense terms for accurate alignment. Observing that motion blur is a ubiquitous artifact in many video frames from the input, we select a single candidate frame for every local region of the 3D surface in order to balance the sharpness and boundary coherency.
From another perspective, we propose to learn a deep metric that tolerates these errors instead of removing them and jointly optimize it with the texture so that the learned metric guides the direction of the realistic textures optimization.

While an image-based convolution operator is an intermediate solution for texture feature extraction, we further explore the convolution directly applied in the texture space of the 3D surface.
%
We observe that the key challenge is to define a consistent canonical 2D parameterization space of the 3D surface for such a convolution operator to apply.
%
This challenge is related to the seamless surface parameterization problem in the computational geometry community. While existing state-of-the-art methods tread it as a mixed-integer programming problem which is NP-hard to solve, we derive a simplified version which reduces the problem into a minimum cost flow problem with polynomial time complexity. Therefore, we achieve a quadrangulation algorithm called \emph{Quadriflow} which is robust and scalable.

Based on the parameterization, we propose \emph{TextureNet} as a neural network formed with a 2D convolution operator in the local tangent space given our canonical parametrization.
%
Our 2D convolution is efficient and handles high-resolution texture signals, which outperforms other less efficient dense 3D convolution operators in the task of 3D semantics understanding.

Finally, the geometric surface parametrization serves as not only a basis for 3D convolution operators to apply but also useful information that is highly-correlated and learnable from the color signals. With the existence of scanning data with aligned RGB images, we propose to render the pre-computed 3D canonical frames given the camera transformation to the input views and train a neural network (\emph{FrameNet}) to estimate them from the RGB images. With understanding the 3D canonical frames from RGB images, our network enables applications ranging from surface normal estimation, feature matching and augmented reality. 

In the future, several challenging problems deserve to be solved. First, the texture is not independent of the geometry, and we believe it is a promising direction to jointly enhance the surface texture and geometry quality altogether, and a joint representation and optimization of these two different types of data are interesting directions to explore. Second, the understanding and the reconstruction of the textures can further benefit each other. One direction is to introduce our texture space convolutions into the process of reconstruction using adversarial texture optimization. Another direction is to view the reconstruction and the learning tasks as a joint problem that can be optimized in an end-to-end manner.